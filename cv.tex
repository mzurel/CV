\documentclass[letterpaper,11pt]{article}

\usepackage[empty]{fullpage}
\usepackage[margin=1.5cm]{geometry}
\raggedright
\raggedbottom

\newcommand{\printable}{0}

\usepackage[x11names]{xcolor}
\usepackage[pdftex,
colorlinks=true,
linkcolor=black,
urlcolor=Blue4,
citecolor=black,
allbordercolors={0.8 0.8 0.8}]{hyperref}

\usepackage{amsmath,amsfonts,amssymb,physics}

\usepackage[maxnames=5,backend=biber,citestyle=numeric-comp,bibstyle=numeric,sorting=none,sortcites=true]{biblatex}
\addbibresource{biblio.bib}

% Format section headings
\usepackage{titlesec}
\titleformat{\section}{
	\vspace{-4pt}\scshape\raggedright\large
}{}{0em}{}[\color{black}\titlerule \vspace{-5pt}]

% Tabular formatting
\usepackage{enumitem}
\renewcommand{\labelitemii}{$\circ$}

\newcommand{\EducationItem}[4]{
	\item{\vspace{-1pt}
		\begin{tabular*}{0.97\textwidth}{l@{\extracolsep{\fill}}r}
			\textbf{#1} & {#2}\\
			{#3} & {#4}
		\end{tabular*}
		\vspace{-5pt}
	}
}
\newcommand{\EducationSubItem}[2]{
	\item{\vspace{-1pt}
		\begin{tabular*}{0.92\textwidth}{l@{\extracolsep{\fill}}r}
			{\footnotesize #1} & {\footnotesize #2}
		\end{tabular*}
		\vspace{-5pt}
	}
}

\newcommand{\ExperienceItem}[4]{
	\item{\vspace{-1pt}
		\begin{tabular*}{0.97\textwidth}{l@{\extracolsep{\fill}}r}
			\textbf{#1} & {#2}\\
			{#3} & {#4}
		\end{tabular*}
		\vspace{-5pt}
	}
}
\newcommand{\ExperienceSubItem}[2]{
	\item{\vspace{-3pt}
		\begin{tabular*}{0.92\textwidth}{l@{\extracolsep{\fill}}r}
			{\footnotesize #1} & {\footnotesize #2}
		\end{tabular*}
		\vspace{-5pt}
	}
}

% Conference item data
% 1: Conference name
% 2: Format
% 3: Title
\newcommand{\ConferenceItem}[3]{
	\item{\parbox{0.97\textwidth}{
		{#1}  \hfill {#2}\\
		\emph{#3}
	}}
}

\newcommand{\OrganizerItem}[4]{
	\item{\parbox{0.97\textwidth}{
			{#1} \hfill {#2}\\
			\emph{#3} \hfill {#4}
		}}
}

\newcommand{\TeachingItem}[2]{
	\item{\vspace{-1pt}
		\begin{tabular*}{0.92\textwidth}{l@{\extracolsep{\fill}}r}
			{#1} & {#2}
		\end{tabular*}
		\vspace{-5pt}
	}
}

\newcommand{\AwardsItem}[2]{
	\item{\vspace{-1pt}
		\begin{tabular*}{0.97\textwidth}{l@{\extracolsep{\fill}}r}
			{#1} & {#2}
		\end{tabular*}
		\vspace{-5pt}
	}
}

\newcommand{\SkillsItem}[2]{
	\item{\vspace{-1pt}
		\begin{tabular*}{0.97\textwidth}{l l}
			{#1:} & {#2}
		\end{tabular*}
		\vspace{-5pt}
	}
}

\begin{document}

\begin{tabular*}{\textwidth}{l@{\extracolsep{\fill}}r}
	\parbox{0.5\textwidth}{
		{\Huge \textbf{Michael Zurel}}
	}
	&
	\begin{tabular}{l}
		Personal email: \href{mailto:mzurel@protonmail.com}{mzurel@protonmail.com}\\
		UBC email: \href{mailto:mzurel@phas.ubc.ca}{mzurel@phas.ubc.ca}\\
		Website: \href{https://mzurel.github.io}{mzurel.github.io}\\
		%Github: \href{https://github.com/mzurel}{github.com/mzurel}
	\end{tabular}
\end{tabular*}


\section*{Education}
\begin{itemize}[leftmargin=*]
	\EducationItem{PhD (in progress)}{University of British Columbia}{Physics (Quantum information and computation)}{Sep., 2020 -- Jul., 2024}
	\begin{itemize}
		\vspace{-4pt}
		\EducationSubItem{Supervisors: Dr.~Robert Raussendorf and Dr.~William G. Unruh}{}
		\EducationSubItem{Affiliations: University of British Columbia (Physics) \& Stewart Blusson Quantum Matter Institute}{}
		\EducationSubItem{Thesis: Classical descriptions of quantum computations}{}
	\end{itemize}
	
	\EducationItem{MSc}{University of British Columbia}{Physics (Quantum information and computation)}{Sep., 2019 -- Nov., 2020}
	\begin{itemize}
		\vspace{-4pt}
		\EducationSubItem{Supervisor: Dr.~Robert Raussendorf}{}
		\EducationSubItem{Thesis: \href{https://dx.doi.org/10.14288/1.0394790}{Hidden variable models and classical simulation algorithms for quantum computation with magic states on qubits}}{}
	\end{itemize}
	
	\EducationItem{BSc}{University of British Columbia}{Combined honours in Physics and Mathematics}{Sep., 2014 -- May, 2019}
	\begin{itemize}
		\vspace{-4pt}
		\EducationSubItem{Honours thesis: Contextuality and Simulating Quantum Computation with Magic States}{}
	\end{itemize}
	%\EducationItem{Memorial University}{St. John’s, Newfoundland}{Completed two physics courses during final year of high school}{2013-2014}
	%\EducationItem{International Summer School for Young Physicists}{Waterloo, Ontario}{Two week summer program hosted by the Perimeter Institute for Theoretical Physics}{2014}
	%\EducationItem{Gonzaga High School}{St. John’s, Newfoundland}{Graduated with First Class Honours, French Immersion Designation}{2012-2014}
\end{itemize}


%%%%%%%%%%%%%%
% Experience %
%%%%%%%%%%%%%%
\section*{Experience}
\begin{itemize}[leftmargin=*]
	\ExperienceItem{Research assistant}{Vancouver, Canada}{Department of Physics \& Astronomy, University of British Columbia}{Sep., 2018 -- Present}
	
	\ExperienceItem{Teaching assistant}{Vancouver, Canada}{Department of Physics \& Astronomy, University of British Columbia}{Sep., 2019 -- Dec., 2023}
	
	\ExperienceItem{Research assistant}{St.~John's, Canada}{Department of Geography, Memorial University}{May, 2017 -- Aug., 2018}
	%\begin{itemize}
	%	\vspace{-4pt}
	%	\ExperienceSubItem{Data acquisition, database management, data analysis (primarily census and socio-economic data)}{}
	%	\ExperienceSubItem{Creation of dynamic data visualizations}{}
	%\end{itemize}
	%\ExperienceItem{Litter Collection Crew}{St. John's, Canada}{City of St. John’s}{May, 2015 -- August, 2015}
	%\ExperienceItem{Volunteer hockey coach}{St. John's, Canada}{Avalon Minor Hockey Association}{September, 2007 -- April, 2014}
	%\ExperienceItem{President of Junior Achievement company}{St. John's, Canada}{Junior Achievement}{2012 -- 2013}
\end{itemize}

% Publications section
\nocite{ZurelHeimendahl2024,ZurelCohenRaussendorf2023,ZurelRaussendorf2023,RaussendorfFeldmann2023,OkayRaussendorf2021,Zurel2020,ZurelRaussendorf2020,RaussendorfZurel2020}
\printbibliography[title={Publications \& preprints}]
For PDFs see \href{mzurel.github.io}{mzurel.github.io}; for citation data see \href{https://scholar.google.com/citations?user=qUA_szUAAAAJ&hl=en&oi=ao}{Google Scholar}.


\section*{Patents}
\begin{itemize}[leftmargin=*]
	\ConferenceItem{Patent Application US20230206102A1; EP4128083A1; WO2021195783A1}{Status: Pending}{Method of simulating a quantum computation, system for simulating a quantum computation, method for issuing a computational key, system for issuing a computational key}
\end{itemize}


\section*{Software}
\begin{itemize}[leftmargin=*]
	\item \href{https://github.com/mzurel/BinarySymplectic.jl}{BinarySymplectic.jl} --- Tools for working with symplectic vector spaces and symplectic groups over $\mathbb{Z}_2$.
	\vspace{-4pt}
	\item \href{https://github.com/mzurel/RandomQM.jl}{RandomQM.jl} --- Julia functions for generating random quantum states and random quantum channels
	\vspace{-4pt}
	\item \href{https://github.com/mzurel/RandomStabilizers.jl}{RandomStabilizers.jl} --- Julia code for generating random stabilizer states and random symplectic group elements based on the ``SYMPLECTICImproved'' algorithm of J.~Math.~Phys.~\textbf{55}~122202~(2014).
	\vspace{-4pt}
	%\vspace{-4pt}
	%\item FiniteSymplectic.jl --- Tools for working with symplectic modules and symplectic groups over $\mathbb{Z}_d$.
	\item NetworkViz --- Data visualization web app for input-output data, census data, and other socio-economic data in Newfoundland and Labrador.  Written in R.
	\vspace{-4pt}
\end{itemize}
Code available on GitHub: \href{https://github.com/mzurel}{github.com/mzurel}

%% Conference section
\section*{Conference talks and seminars}
\begin{itemize}[leftmargin=*]
	\ConferenceItem{Quantum Physics and Logic (QPL), Jul., 2023}{}{Efficient classical simulation of quantum computation beyond Wigner positivity}
	
	\ConferenceItem{Southwest Quantum Information and Technology (SQuInT) Workshop, Oct., 2023}{30 minute talk}{Simulating quantum computation: how many ``bits'' for ``it''?}
	
	\ConferenceItem{QLOC Group Seminar @ Iberian Nanotechnology Laboratory, Sep., 2023}{60 minute talk}{A hierarchy of classical simulation algorithms for quantum computation}
	
	\ConferenceItem{Quantum Physics and Logic (QPL), Jul., 2023 (presented by a co-author)}{30 minute talk}{Simulation of quantum computation with magic states via Jordan-Wigner transformations}
	
	\ConferenceItem{Coogee 2023 Workshop, Feb., 2023}{60 minute talk}{No-go theorems for discrete Wigner functions and alternative quasiprobability representations for quantum computation with magic states}
	
	\ConferenceItem{Shealf talks (Samson Abramsky group seminar @ University of Oxford), Dec., 2022}{60 minute talk}{The role of cohomology in quantum computation with magic states}
	
	\ConferenceItem{``FoQaCiA'' collaboration kick-off meeting, Nov., 2022}{60 minute talk}{$\Lambda$ polytopes and classical simulation of quantum computation with magic states}
	
	\ConferenceItem{Theory of Quantum Computation, Communication, and Cryptography (TQC), Jul., 2022}{25 minute talk}{Hidden Variable Model for Quantum Computation with Magic States on Qudits of Any Dimension}
	
	\ConferenceItem{David Gross group seminar @ University of Cologne, Jul., 2022}{60 minute talk}{Quasiprobability representations for quantum computation with magic states}
	
	\ConferenceItem{Quantum Physics and Logic (QPL), Jun., 2022}{10 minute talk}{Hidden Variable Model for Quantum Computation with Magic States on Qudits of Any Dimension}
	
	\ConferenceItem{Bilkent University Math Grad Seminar, Jun., 2022}{60 minute talk}{Polytopes in quantum computation and quantum information}
	
	\ConferenceItem{Algebraic Structures in Quantum Computation V (ASQC5), Jun., 2022}{45 minute talk}{Hidden variable models for quantum computation with magic states}
	
	\ConferenceItem{UBC Institute of Applied Mathematics Grad Seminar, Jun., 2022}{60 minute talk}{Polytopes in quantum computation and quantum information}
	
	\ConferenceItem{Internal talk for QuEra Computing Inc. software/algorithms team, Apr., 2022}{45 minute talk}{Classical simulation of quantum computation with magic states}
	
	\ConferenceItem{Theory of Quantum Computation, Communication, and Cryptography (TQC), Jul., 2021}{30 minute talk}{Hidden variable model for universal quantum computation with magic states on qubits}
	
	\ConferenceItem{Quantum Physics and Logic (QPL), Jun., 2021}{30 minute talk}{Hidden variable model for universal quantum computation with magic states on qubits}
	
	\ConferenceItem{Algebraic Structures in Quantum Computation IV (ASQC4), Jun., 2020}{60 minute talk}{Hidden variable model for universal quantum computation with magic states on qubits}
	
	\ConferenceItem{Quantum Physics and Logic (QPL), Jun., 2019}{25 minute talk}{Phase-space-simulation method for quantum computation with magic states on qubits}
\end{itemize}
For slides, videos, etc., see \href{https://mzurel.github.io/talks}{mzurel.github.io/talks}

\section*{Poster presentations}
\begin{itemize}[leftmargin=*]
	\ConferenceItem{Theory of Quantum Computation, Communication, and Cryptography (TQC), Sep., 2024}{}{Efficient classical simulation of quantum computation beyond Wigner positivity}
	
	\ConferenceItem{Quantum Information Processing (QIP), Jan., 2024}{}{Simulation of quantum computation with magic states via Jordan-Wigner transformations}
	
	\ConferenceItem{Quantum Information Processing (QIP), Jan., 2024}{}{Simulating quantum computation: how many ``bits'' for ``it''?}
	
	\ConferenceItem{Southwest Quantum Information and Technology (SQuInT) Workshop, Oct., 2023}{}{Simulation of quantum computation with magic states via Jordan-Wigner transformations}
	
	\ConferenceItem{Quantum Physics and Logic (QPL), Jul., 2023}{}{Simulating quantum computation with magic states: how many ``bits'' for ``it''?}
	
	\ConferenceItem{Max Planck - UBC - UTokyo Centre for Quantum Materials Annual Meeting, Sep., 2022}{}{Hidden variable model for quantum computation with magic states on qudits of any dimension}
	
	\ConferenceItem{Max Planck - UBC - UTokyo Centre for Quantum Materials Annual Meeting, Sep., 2022}{}{The role of cohomology in quantum computation with magic states}
	
	\ConferenceItem{Theory of Quantum Computation, Communication, and Cryptography (TQC), Jul., 2022}{}{The role of cohomology in quantum computation with magic states}
	
	\ConferenceItem{Quantum Information Processing (QIP), Mar., 2022}{}{Hidden Variable Model for Quantum Computation with Magic States on Any Number of Qudits of Any Dimension}
	
	\ConferenceItem{Quantum Information Processing (QIP), Mar., 2021}{}{Hidden variable model for universal quantum computation with magic states on qubits}
	
	\ConferenceItem{Southwest Quantum Information and Technology (SQuInT), Feb., 2020}{}{Phase-space-simulation method for quantum computation with magic states on qubits}
\end{itemize}
For poster PDFs see \href{https://mzurel.github.io/talks}{mzurel.github.io/talks}

\section*{Workshop \& summer school organization}
\begin{itemize}[leftmargin=*]
	\OrganizerItem{Summer School on the Foundations of Quantum Computational Advantage}{July, 2023 (postponed to 2024)}{Bilkent University, Ankara, Turkey}{Project mentor}
	\OrganizerItem{Algebraic Structures in Quantum Computation V (ASQC5)}{June, 2022}{University of British Columbia, Vancouver, Canada}{Co-organizer}
	\OrganizerItem{Cornerstone Models of Quantum Computing Summer School}{August, 2021}{TRIUMF, Vancouver, Canada}{Teaching assistant for MBQC section}
	\OrganizerItem{Cornerstone Models of Quantum Computing Summer School}{August, 2020}{TRIUMF, Vancouver, Canada}{Teaching assistant for MBQC section}
\end{itemize}

% Awards section
\section*{Awards}
\begin{itemize}[leftmargin=*]
	\AwardsItem{NSERC Postdoctoral Fellowship (NSERC PDF)}{2024 -- 2026}
	\AwardsItem{CGS - Michael Smith Foreign Study Supplement (NSERC CGS-MSFSS)}{2023}
	\AwardsItem{Alexander Graham Bell Canada Graduate Scholarship (NSERC CGS-D)}{2021 -- 2024}
	\AwardsItem{UBC Four Year Doctoral Fellowship (4YF)}{2021 -- 2025}
	\AwardsItem{President's Academic Excellence Initiative PhD Award}{2020 -- 2024}
	\AwardsItem{UBC Faculty of Science PhD Tuition Award }{2020 -- 2024}
	%\AwardsItem{UBC Chancellor's Scholar Award}{2014}
	%\AwardsItem{The Duke of Edinburgh's International Award Gold Award Received}{2014}
	%\AwardsItem{CAP High School Physics Examination - $1^{st}$ Place in Newfoundland \& Labrador}{2014}
	%\AwardsItem{Gonzaga High School Award of Excellence}{2012 -- 2014}
	%\AwardsItem{AAA hockey provincial champion}{2012 -- 2014}
	%\AwardsItem{Newfoundland provincial lacrosse team}{2011 -- 2013}
\end{itemize}

\section*{Peer review}
Referee for the following journals:\vspace{-2mm}
\begin{itemize}[leftmargin=*]
	\AwardsItem{Physical Review Letters}{}
	\AwardsItem{PRX Quantum}{}
	\AwardsItem{Physical Review A}{}
	\AwardsItem{Quantum Journal}{}
	\AwardsItem{Journal of Physics A: Mathematical and Theoretical}{}
\end{itemize}
Referee for the following conferences:\vspace{-2mm}
\begin{itemize}[leftmargin=*]
	\AwardsItem{Quantum Information Processing (QIP)}{}
\end{itemize}


% Technical skills section
\section*{Technical Skills}
\begin{itemize}[leftmargin=*]
	\SkillsItem{Programming languages}{Python, Julia, Matlab, R, SQL}
	\SkillsItem{Technologies}{Linux, Latex, Git, AWS, MariaDB}
\end{itemize}

% Teaching section
\section*{Teaching experience}
\begin{itemize}[leftmargin=*]
	\TeachingItem{Teaching assistant: Computational Physics}{Sep., 2023 -- Dec., 2023}
	\TeachingItem{Teaching assistant: Frontiers in Physics}{Sep., 2023 -- Dec., 2023}
	\TeachingItem{Teaching assistant: Introduction to Quantum Mechanics}{Jan., 2022 -- Apr., 2022}
	\TeachingItem{Teaching assistant: Electricity and Magnetism}{Sep., 2021 -- Dec., 2021}
	\TeachingItem{Teaching assistant: Electricity and Magnetism}{Sep., 2020 -- Dec., 2020}
	\TeachingItem{Teaching assistant: Enriched Physics I}{Sep., 2020 -- Dec., 2020}
	\TeachingItem{Teaching assistant: Introductory Physics for Engineers II}{Jan., 2020 -- Apr., 2020}
	\TeachingItem{Teaching assistant: Introductory Physics}{Sep., 2019 -- Dec., 2019}
\end{itemize}

\vspace{1cm}

\begin{flushright}
	Last updated: May, 2024
\end{flushright}

\end{document}